\documentclass{article}

\usepackage{amssymb}
\usepackage{amsmath}
\usepackage{amsthm}
\usepackage{enumitem}

\usepackage{float}
\usepackage{xcolor, graphicx}
\usepackage{hyperref}

\usepackage{soul}

\setlength{\textwidth}{7in}
\setlength{\evensidemargin}{-0.24in}
\setlength{\oddsidemargin}{-0.24in}
\setlength{\textheight}{8.45in}
\setlength{\topmargin}{-0.45in}
\setlength{\parindent}{0.3in}
\headheight72pt
\headsep22pt

\pagestyle{myheadings}
\usepackage{fancyhdr}
\renewcommand{\headrulewidth}{0pt}

\pagestyle{fancy}
\fancyhf{}



\pagestyle{fancy}
\fancyhf{}
\lhead{\textbf{Derek Yu 101331395 Bonus MCQ 1}\vspace{-5pt}\\\hrulefill}
\fancyfoot[C]{\thepage}
\def\qed{\ \ \vrule height6pt width5pt depth3pt}
\renewcommand{\implies}{\rightarrow}
\newcommand{\xor}{\oplus}
\newcommand{\same}{\leftrightarrow}
\newcommand{\ov}[1]{\overline{#1}}

% Blue solution
\newcommand{\sol}[1]{\textbf{Solution:\,}\textcolor{blue}{#1}}


\begin{document}
\bibliographystyle{alpha}
\title{COMP 1805 - Bonus 1}
\author{Derek Yu}
\date{} %comment this out if you want the date to be shown.
\maketitle
\thispagestyle{fancy}

\textbf{Propositional Logic [1-4]}
\begin{enumerate}

\item Which proposition is equivalent to the following: $a\land (\neg b\lor(c\implies d))$
\begin{enumerate}
\item \hl{$\neg(\neg a\lor (b\land c\land \neg d))$}
\item $\neg(\neg a\land (b\lor c\land d))$
\item $\neg(\neg a\implies (b\land c)\lor \neg d)$
\item $\neg (a\lor (b\land c\lor d))$
\end{enumerate}

\item If $b=$"I will ball", $t=$"I will take the table", $p=$"I will take the piano", convert the following into a logical expression: I will ball if I take the table or piano.
\begin{enumerate}
\item {$(t\land p)\implies b$}
\item {$(t\lor p)\leftrightarrow b$}
\item \hl{$(t\lor p)\implies b$}
\item {$b\implies(t\land p)$}
\end{enumerate}

\item Find the logically equivalent statement for the following: I won a prize and didn't pay my fee.
\begin{enumerate}
\item It is false that I didn't win a prize or I paid my fee.
\item It is true that I won a prize, it is true that I paid my fee.
\item \hl{It is false that I either didn't win a prize or paid my fee.}
\item It is true that I didn't win a prize and I paid my fee.
\end{enumerate}

\item Which statement is not a proposition?
\begin{enumerate}
\item The apple in bed is void.
\item Creating a bus driver removes levels of conundrum.
\item \hl{Take a deep breath to appreciate that the sky is green.}
\item Huge wins are for those who are unemployed.
\end{enumerate}

\newpage

\textbf{Predicate Logic [5-8]}

\item $U=$``All Humans''
\\$A(x)=$``$x$ has an apple''
\\$B(x)=$``$x$ beat Michael Jordan''
\\$C(x)=$``$x$ owns an apartment''\\ 
\\Convert the following into logic:
\\A person who beat Michael Jordan must have an apple and can't own an apartment.
\begin{enumerate}
\item $\forall x(B(x)\land A(x)\land\neg C(x))$
\item $\forall x(B(x)\leftrightarrow (A(x)\land\neg C(x)))$
\item \hl{$\forall x(B(x)\implies (A(x)\land\neg C(x)))$}
\item $\forall x(B(x)\lor\neg B(x)\land (A(x)\land\neg C(x)))$
\end{enumerate}

\item Find the correct negation of the following expression: $\forall x(A(x)\leftrightarrow\neg B(x))$
\begin{enumerate}
\item \hl{$\exists x((A(x)\land B(x))\lor(\neg A(x)\land\neg B(x)))$}
\item $\exists x\neg ((A(x)\lor B(x))\land (\neg A(x)\lor B(x)))$
\item $\forall x((B(x)\land\neg A(x))\lor (A(x)\land B(x)))$
\item $\forall x((A(x)\land B(x))\lor(\neg A(x)\land\neg B(x)))$
\end{enumerate}

\item $U=$``All Humans''
\\$T(x)=$``$x$ takes the train''
\\$N(x)=$``$x$ negates the noodle house''
\\$W(x)=$``$x$ walks the dog''\\ 
\\Convert the following into English: $\exists x(W(x)\land (N(x)\lor T(x)))$
\begin{enumerate}
\item Everyone who walks the dog but either negates the noodle house or takes the train.
\item There exists someone who walks the dog and negates the noodle house or takes the train.
\item Everyone walks the dog if they negate the noodle house or takes the train.
\item \hl{There exists someone who walks the dog and either negates the noodle house or takes the train.}
\end{enumerate}

\item For what domain is the following expression True: $\forall x((3<x\leq18)\land (\frac{x}{3}=0))$
\begin{enumerate}
\item $U=\{3,6, 9, 12, 15, 18\}$
\item \hl{$U=\{6, 9, 12, 15, 18\}$}
\item $U=\{4, 6, 8, 10,12, 14,16, 18\}$
\item $U=\{6, 9, 12, 15\}$
\item $U=\{3,18\}$
\item $U=\{6, 9, 12, 15,18\}$
\end{enumerate}

\newpage

\textbf{Validity of Logical Arguments [9-11]}

\item 
\begin{enumerate}[label =(\arabic*), ref = \arabic*]
	\item $a\implies b$
	\item $c\leftrightarrow b$
	\item $\neg c\land d$ \hspace{50pt}\vspace{3pt}
	\vspace{5pt} \hrule \vspace{5pt}
	Given the premises above, we can conclude:\\$\therefore $ ?\hfill $\square$
\end{enumerate} 
\begin{enumerate}
\item $d=False$
\item \hl{$a=False$}
\item $e=True$
\item $b=True$
\end{enumerate}

\item Select the invalid argument.
\begin{enumerate}
\item \hl{$(a\land b), (c\implies \neg b)$}\hl{\quad$\therefore c$}
\item $(a\lor b\lor c), (\neg b\implies \neg c), c$\quad$\therefore b$
\item $(a\land(b\lor c)), (a\lor\neg b)$\quad$\therefore c\lor\neg c$
\item $(a\lor b\lor c), (\neg b\implies \neg c), \neg(a\lor c)$\quad$\therefore b$
\end{enumerate}

\item Select the correct statement about the argument below:
\begin{enumerate}[label =(\arabic*), ref = \arabic*]
	\item $b\lor a$
	\item $d\land b$
	\item $b\leftrightarrow c$
	\item $e\lor d$
	\item $\neg c\implies b$ \hspace{50pt}\vspace{3pt}
	\vspace{5pt} \hrule \vspace{5pt}
\end{enumerate} Options:
\begin{enumerate}
\item This argument is invalid.
\item \hl{This argument is a tautology.}
\item This argument is a contingency.
\item This argument is a contradiction.
\end{enumerate}

\newpage

\textbf{Validity \& Quantifiers [12-14]}
\\$U=$``All Humans''
\\$T(x)=$``$x$ takes the train''
\\$N(x)=$``$x$ negates the noodle house''
\\$W(x)=$``$x$ walks the dog''\\ 
\item Find the invalid argument:
\begin{enumerate}

\item 
\begin{enumerate}[label =(\arabic*), ref = \arabic*]
	\item $\forall x(T(x)\implies N(x))$
	\item $\exists x(\neg N(x))$
\end{enumerate} 
$\quad \therefore \exists x(\neg T(x))$\vspace{3pt}

\item 
\begin{enumerate}[label =(\arabic*), ref = \arabic*]
	\item \protect\hl{$\exists x(T(x)\lor N(x))$}
	\item \protect\hl{$\exists x(\neg N(x))$}
\end{enumerate} 
$\quad \therefore $ \protect\hl{$\exists x(T(x)\land\neg N(x))$}\vspace{3pt}

\item 
\begin{enumerate}[label =(\arabic*), ref = \arabic*]
	\item $\forall x(T(x)\land N(x))$
	\item $\exists x(T(x)\implies N(x))$
\end{enumerate} 
$\quad \therefore \exists x(T(x)\lor N(x))$\vspace{3pt}

\item 
\begin{enumerate}[label =(\arabic*), ref = \arabic*]
	\item $\neg\exists x(T(x)\implies N(x))$
	\item $\exists x(\neg N(x))$
\end{enumerate} 
$\quad \therefore \forall x(T(x)\land \neg N(x))$\vspace{3pt}

\end{enumerate}

\item Select the most appropriate rule of inference to be applied at ``$?$'' in the argument below:
\begin{enumerate}[label =(\arabic*), ref = \arabic*]
	\item $\forall x(W(x)\lor T(x))$
	\item $?$
\end{enumerate} 
$\quad \therefore\exists x(T(x))$\vspace{3pt}\\
Options:
\begin{enumerate}
\item \hl{Universal Instantiation}
\item Universal Generalization
\item Existential Instantiation
\item Existential Generalization
\end{enumerate}

\item Select the correct set of elements for the argument below:
\begin{enumerate}[label =(\arabic*), ref = \arabic*]
	\item $\exists (x,y)(W(x)\land T(y))$
	\item $\exists (x,z)(N(z)\land ((z\neq x)\implies \neg N(z)))$
\end{enumerate} 
Options:
\begin{enumerate}
\item $x=John$, $y=Jane$, $z=Jane$
\item $x=John$, $y=Jill$, $z=Jane$
\item $x=Jane$, $y=Jane$, $z=John$
\item \hl{$x=John$, $y=Jane$, $z=John$}
\end{enumerate}

\newpage

\textbf{Proof Techniques [15-17]}

\item Find the mistake in the following proof:\\\\
Prove that if integer $a$ is greater than 1, then $a$ is is not equal to 0.
\\Indirect proof: Prove that if integer $a$ is equal to 0, then $a$ is smaller than 1.
\\Assume $a$ is an integer equal to 0; then $a=0.$
\\$0<1$\quad $\therefore a<1$\hfill $\square$
\begin{enumerate}
\item Circular reasoning
\item \hl{Incorrect premise}
\item Lack of evidence in arguments
\item Does not prove the original statement
\end{enumerate}

\item What proof technique is being used in the example below:\\
Theorem: "If $a$ is an integer greater than integer $b$, then $\frac{a}{2}<\frac{b}{2}$".\\
Let $a=2, b=1$. Then $a>b$ since $2>1$;\\
$(\frac{a}{2}<\frac{b}{2})=(\frac{2}{2}<\frac{1}{2})=(1<1/2)$\quad $\therefore$ This statement is False.
\hfill $\square$
\begin{enumerate}
\item Contradiction
\item Proof by case
\item \hl{Counterexample}
\item Direct Proof
\end{enumerate}

\item Which proof technique would be most effective to prove the following theorem:\\
``If integer $a$ is negative and integer $b$ is positive, then $|a||b|$ must be positive.''
\begin{enumerate}
\item Direct Proof
\item Indirect Proof
\item Proof by Contradiction
\item Proof by Cases
\item \hl{Trivial Proof}
\item Pigeonhole Principle
\item Existence Proof
\end{enumerate}

\newpage

\textbf{Set Theory [18-20]}\\\\
Given the following sets:\\
$A=\{a,b,c,d,e,f\}$\\
$B=\{b,d,e,g,f\}$\\
$C=\{d,o,n,m\}$\\
\item Given:\\
$F=(A\cap B)\setminus C$\\
What is the set of $F$?
\begin{enumerate}
\item $F=\{a,b,c,r,g,f\}$
\item $F=\{a,b,c,d, r,g,f\}$
\item \hl{F=$\{b,e,f\}$}
\item $F=\{b,d,e,f\}$
\item $F=\{a\}$
\item $F=\{b,e,f,g\}$
\item $F=\{a,d,g,o,n,m\}$
\end{enumerate}

\item Which statement about the sets is false?
\begin{enumerate}
\item \hl{$\{b,d,e\}\subseteq (A\cap B)$}
\item $\{d\}\subset(A\cap B\cap C)$
\item $(C\setminus B)\subseteq(C)$
\item $\{b,e,\emptyset\}\subset(A)$
\end{enumerate}

\item Which is the correct power set of $D=\{1,2,3\}$?
\begin{enumerate}
\item $P(D)=\{\emptyset,\{1\},\{2\},\{3\},\{1,2\},\{2,1\},\{2,3\},\{3,2\},\{1,3\},\{3,1\},\{1,2,3\},\{3,2,1\},\{2,3,1\}\}$
\item $P(D)=\{\emptyset,\{1\},\{2\},\{3\},\{1,2,3\}\}$
\item $P(D)=\{\{1\},\{2\},\{3\},\{1,2\},\{2,3\},\{1,3\},\{1,2,3\}\}$
\item $P(D)=\{\{1\},\{2\},\{3\},\{1,2\},\{2,1\},\{2,3\},\{3,2\},\{1,3\},\{3,1\},\{1,2,3\},\{3,2,1\},\{2,3,1\}\}$
\item \hl{$P(D)=\{\emptyset,\{1\},\{2\},\{3\},\{1,2\},\{2,3\},\{1,3\},\{1,2,3\}\}$}
\end{enumerate}

\end{enumerate}
\end{document}
