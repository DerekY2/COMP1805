\documentclass{article}

\usepackage{amssymb}
\usepackage{amsmath}
\usepackage{amsthm}
\usepackage{enumitem}

\usepackage{float}
\usepackage{xcolor, graphicx}
\usepackage{hyperref}

\setlength{\textwidth}{7in}
\setlength{\evensidemargin}{-0.24in}
\setlength{\oddsidemargin}{-0.24in}
\setlength{\textheight}{8.45in}
\setlength{\topmargin}{-0.45in}
\setlength{\parindent}{0.3in}
\headheight72pt
\headsep22pt

\pagestyle{myheadings}
\usepackage{fancyhdr}
\renewcommand{\headrulewidth}{0pt}

\pagestyle{fancy}
\fancyhf{}



\pagestyle{fancy}
\fancyhf{}
\lhead{\textbf{Your Name and student ID}\vspace{-5pt}\\\hrulefill}

\def\qed{\ \ \vrule height6pt width5pt depth3pt}
\renewcommand{\implies}{\rightarrow}
\newcommand{\xor}{\oplus}
\newcommand{\same}{\leftrightarrow}
\newcommand{\ov}[1]{\overline{#1}}

% Blue solution
\newcommand{\sol}[1]{\textbf{Solution:\,}\textcolor{blue}{#1}}


\begin{document}
\bibliographystyle{alpha}
\title{Assignment 3 - COMP 1805}
\date{} %comment this out if you want the date to be shown.
\maketitle
\thispagestyle{fancy}



\medskip
\begin{enumerate}



\item(8pts) Define each set using set-builder notation. If the set is finite, state its cardinality; otherwise, indicate that it is infinite.

\begin{enumerate}
\item $\{ 10, 20, 30, 40,  \ldots, 1000 \}$
\item $\{ -6, -4, -2, 0, 2, 4, 6 \}$
\item $\{ 0,3,6,9,12, \ldots \}$
\item $\{ 0.5, 1, 2, 4, 8, 16, \ldots \}$
\end{enumerate}




\item(16pts) Determine whether each statement is true or false for all sets $A$ and $B$. Provide a justification for your answer. Note that the difference between two sets $A$ and $B$ can be denoted $A \setminus B$ or $A - B$. 

\begin{enumerate}
\item If $A \subset B$ then $A \subseteq B$.
\item If $A \subseteq B$ then $A \subset B$.
\item If $A = B$ then $A \subseteq B$.
\item If $A = B$ then $B \subset A$.
\item If $A \subset B$ then $A \neq B$ and $B \neq \emptyset$.
\item If $A \subseteq B$ then $A \cup B = B$ and $A \cap B = A$.
\item If $A \cap (B - A) = \emptyset$ then $A = \emptyset$.
\item If $A - (B - A) = \emptyset$ then $A = \emptyset$.
\end{enumerate}
 



\item(10pts) Using the provided template, draw a Venn Diagram to represent the following sets:\\
\begin{center}
\includegraphics[width = 0.27\textwidth]{template.png}
\end{center}

\begin{enumerate} 
\item $(A\cup B) \cap C$ 
\item $A\cup (B \cap C)$ 
\item $\overline{(A \cup B)} \cap C$ 
\item $\overline{(A - B)} \cap C$
\item $(A \cap B) \cup (A \cap \overline{C})$
\end{enumerate}




\item(18pts) Let $A$, $B$ and $C$ be sets. Determine the validity of each statement below. Justify your answer using set identities or membership tables. If two sets are not equivalent, provide a counterexample. A sample solution is provided for part (a).
\begin{enumerate}
\item $A \cap (B-C) = (A-C) \cap B$

\sol{The equation is true. We will show this using set identities. 
\begin{align*}
A \cap (B-C)
 &= A \cap (B \cap \overline{C}) & \text{Difference Equivalence}\\
 &= (A \cap \overline{C}) \cap B & \text{Commutative and Associative Laws}\\
 &= (A - C) \cap B & \text{Difference Equivalence} \\
 \end{align*} 
}

\item $(A-\overline{B}) \cup (B-\overline{A}) = (A \cap B)$ 
\item $\overline{(A-B) \cup (B-A)} = \overline{A}\cup B$
\item $((A - B)\cup(A \cap B))\cap((\ov{B\cap \ov{B}}) - A)= \emptyset$
\item $A-(B-C) = (A-B)-C$
\item $(B \cup C)-(A \cup \ov{C}) = \ov{A} \cap C = \ov{A} \cap (A \cup C)$
\item $(A \cup \ov{C})- (B \cup C) = A \cap B \cap \ov{C}$

\end{enumerate}




\item(10pts) Determine whether $f$ is a function or not. Justify your answer. Recall, that $\mathbb{R}$ is the set of all real numbers, $\mathbb{Z}$ is the set of all integers. 

\begin{enumerate}
\item $f:\mathbb{R} \rightarrow \mathbb{R}$, $f(x)=\frac{1}{3-x}$.
\item $f:\mathbb{R} \rightarrow \mathbb{R}$, $f(x)=\pm \sqrt{x^2+5}$.
\item $f:\mathbb{Z} \rightarrow \mathbb{R}$, $f(x)=\sqrt{x^2+8}$.
\item $f:\mathbb{R} \rightarrow \mathbb{R}$, $f(x)=\sqrt{x+1}$.
\item $f:\mathbb{R} \rightarrow \mathbb{Z}$, $f(x)=\lceil x \rceil$. Note that $f$ is the ceiling function.
\end{enumerate}




\item(21pts) For each of the following functions $f:\mathbb{R} \rightarrow \mathbb{R}$ prove or disprove the following:
\begin{itemize}
\item The function is injective (one-to-one).
\item The function is surjective (onto).
\item The function is bijective (both injective and surjective). If the function is a bijection, find its inverse.
\end{itemize}

\begin{enumerate}	
\item $f(x) = \lfloor 3 x \rfloor$. Note that $\lfloor x \rfloor$ is the \textbf{floor} function.
\item $f(x) = 8-3x$
\item $f(x) = |8 - 3 x|$  
\item $f(x) = 4x^3 + 5$
\item $f(x) = 2x^2-3$
\end{enumerate}




\item(9pts) Let $f$ and $g$ both be functions from real numbers to real numbers. Let $f(x) = 3x^2-8$ and $g(x) = x-2$. Define each of the following, simplify, and then evaluate. You need to show your work.
\begin{enumerate}
\item $(f\circ g) (x= -1)$ 
\item $(g\circ f)(x=2)$
\item $((g\circ f)\circ g) (x=1)$ Reuse solution to part (b).
\end{enumerate}




\item (8 pts) Let $f:\mathbb{R} \rightarrow \mathbb{R}$ and $g:\mathbb{R} \rightarrow \mathbb{R}$. Prove or disprove the followings:

\begin{enumerate}
\item if $f$ and $g$ are both bijections, then $f(x)+g(x)$ is also a bijection. 
\item if $f$ is a bijection, then for any real number $c \neq 0$, $c \cdot f(x)$ is also a bijection. 
\end{enumerate}
 
\end{enumerate}
\end{document}


