\documentclass{article}

\usepackage{amssymb}
\usepackage{amsmath}
\usepackage{amsthm}
\usepackage{enumitem}

\usepackage{float}
\usepackage{xcolor, graphicx}
\usepackage{hyperref}

\setlength{\textwidth}{7in}
\setlength{\evensidemargin}{-0.24in}
\setlength{\oddsidemargin}{-0.24in}
\setlength{\textheight}{8.45in}
\setlength{\topmargin}{-0.45in}
\setlength{\parindent}{0.3in}
\headheight72pt
\headsep22pt

\pagestyle{myheadings}
\usepackage{fancyhdr}
\renewcommand{\headrulewidth}{0pt}

\pagestyle{fancy}
\fancyhf{}



\pagestyle{fancy}
\fancyhf{}
\lhead{\textbf{Your Name and student ID}\vspace{-5pt}\\\hrulefill}

\def\qed{\ \ \vrule height6pt width5pt depth3pt}
\renewcommand{\implies}{\rightarrow}
\newcommand{\xor}{\oplus}
\newcommand{\same}{\leftrightarrow}
\newcommand{\ov}[1]{\overline{#1}}

% Blue solution
\newcommand{\sol}[1]{\textbf{Solution:\,}\textcolor{blue}{#1}}


\begin{document}
\bibliographystyle{alpha}
\title{Assignment 4 - COMP 1805}
\date{} %comment this out if you want the date to be shown.
\maketitle
\thispagestyle{fancy}



\medskip
\begin{enumerate}
\item(12pts) A sequence is \textbf{increasing} if for every two consecutive indices, $k$ and $k+1$ in the domain, $a_{k} < a_{k+1}$. A sequence is \textbf{decreasing} if for every two consecutive indices, $k$ and $k+1$ in the domain, $a_{k} > a_{k+1}$.
Using the definitions of arithmetic and geometric progressions, answer the following questions about sequence properties. Justify your answers.
\begin{enumerate}
\item What are the conditions on the common difference $d$ and initial term $a$ that make the resulting arithmetic progression increasing?
\item What are the conditions on the common difference $d$ and initial term $a$ that make the resulting arithmetic progression decreasing?
\item What are the conditions on the common ratio $r$ and initial term $a$ that make the resulting geometric progression increasing?
\item What are the conditions on the common ratio $r$ and initial term $a$ that make the resulting geometric progression decreasing?
\item Can an arithmetic progression be neither decreasing nor increasing? 
\item Can a geometric progression be neither decreasing nor increasing? 
\end{enumerate}

 

\item(12pts) Below you can see lists of integers. For each of these lists, provide an explicit formula $\{a_n\}$ (but not a recurrence relation) that generates the terms of an integer sequence that begins with the given list. Don't forget to specify the possible values of $n$.
\begin{enumerate}
\item $-5, 5, -5, 5, -5, 5, -5, 5, -5,\dots$
\item $6, 11, 16, 21, 26, 31, 36, 41, 46,\dots$
\item $3, 6, 12, 24, 48, 96, 192, 384, 768, \dots$
\item $3, 6, 11, 18, 27, 38, 51, 66, 83, 102, \dots$
\end{enumerate}



\item(6pts) Express the sum of the first $50$ terms of the following series using sigma notation in two different ways:
\begin{enumerate}
\item $3 + 3 + 3 + 3 + \ldots$
\item $6 + 9 + 12 + 15 + 18 +\ldots$
\end{enumerate}


 
\item(12pts) Compute the exact values of the following sums. You need to show each step in your derivation and simplify the final answer. See sample solution in part (d). 

\begin{enumerate}

\item $\sum_{j=79}^{89} 7$
\item $\sum_{i=1}^8 (i-1)$
\item $\sum_{i=1}^n\sum_{k=30}^n (6i-3)$
\item $\sum_{k=1}^n (\frac{6k^2 - 8k}{n})$

\sol{
	\begin{eqnarray*}
		\sum_{k=1}^n \left(\frac{6k^2 - 8k}{n}\right) & = &\sum_{k=1}^n \frac{6k^2}{n} + \sum_{k=1}^n \frac{-8k}{n} \\
		& = &\frac{6}{n} \sum_{k=1}^n k^2 - \frac{8}{n} \sum_{k=1}^n k \\
		& = & \frac{6}{n} \cdot \frac{n(n+1)(2n+1)}{6} -\frac{8}{n} \cdot \frac{n(n+1)}{2} \\
		& = & (n+1)(2n+1) - 4n - 4 \\
		& = & 2n^2 + 3n+1 - 4n - 4\\
		& = & 2n^2 - n - 3
	\end{eqnarray*}
}

\end{enumerate}



\item(13pts)Provide an algorithm (using pseudocode) for each of the following parts. Additionally, indicate the algorithm's runtime using big-$O$ notation.
\begin{enumerate}
\item Describe an algorithm that takes as input a sequence of integers $a_1$, $a_2$, $\ldots$, $a_n$ (for $n\geq 2$) and determines whether any two numbers in the sequence sum to $9$. The algorithm should return ``True'' if such a pair exists and ``False'' otherwise. Ensure that your solution avoids redundant computations, such as rechecking pairs that have already been considered.

\item You are given an array of integers, where each element represents the stock price on a particular day. Write an algorithm that calculates the maximum profit that can be obtained by buying and selling a single share of the stock. Note that you cannot sell a stock before you buy one.\\
For example, given the array $7, 1, 5, 3, 6, 4$, the maximum profit of $5$ can be obtained by buying on day $2$ (price = 1) and selling on day 5 (price = 6).
\end{enumerate}



\item(18pts) Suppose that $f(n)$ is $\Omega(n)$ and $g(n)$ is $O(n^2)$. Prove or disprove the following statements. Note that to disprove a statement, you need to provide a counter-example.
\begin{enumerate}
\item $f(n)$ is $O(n)$.
\item $f(n)$ is $\Omega(1)$.
\item $f(n)$ is $O(n^2)$.
\item $g(n)$ is $\Omega(n)$.
\item $g(n)$ is $\Omega(n^3)$.
\item $g(n)$ is $O(n^3)$.
\end{enumerate}



\item(6pts) Suppose that Algorithm $A$ has runtime complexity $O(n^2)$ and Algorithm $B$ has runtime complexity $O(n \log n)$, where both algorithms solve the same problem.

\begin{enumerate}
\item Assume you started both algorithms simultaneously, and assume that $n=50$. Is it possible for Algorithm $A$ to terminate before Algorithm $B$?
\item Is it possible, that actual running time of $A$ and $B$ is the function $n$?
\item Suppose we also know that Algorithm $A$ has runtime complexity $\Omega(n^2)$, and Algorithm $B$ has runtime complexity $\Omega(n \log n)$. In other words, runtimes of $A$ and $B$ are $\Theta(n^2)$ and $\Theta(n \log n)$, respectively. How do the algorithms compare when $n$ is very large?
\end{enumerate}



\medskip
\item(21pts) Determine whether or not the following are true. If true, then provide a derivation showing why.
If false, then provide a derivation showing why. For your proofs use definitions of asymptotic notations. Notice, in your solutions you are not allowed to use limits.
For convenience, you may assume that the logs are in the base of your choice, but you should specify what base you are using in your derivation.
\begin{enumerate}
\item $3^2$ is $\Theta(1)$.
\item $(2 \cdot 2)^n$ is  $O(2^n)$.
\item $2 \cdot 2^n$ is $O(2^n)$.
\item $3n^2+n-5$ is $\Theta(n^2)$.
\item $(n/5)^3 - 2n^2 - n\log n$ is $\Theta(n^2)$.
\end{enumerate}
 
\end{enumerate}
\end{document}

